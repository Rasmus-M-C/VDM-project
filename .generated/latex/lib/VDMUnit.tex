\documentclass[a4paper]{article}
\usepackage{longtable}
\usepackage[color]{vdmlisting}
\usepackage{fullpage}
\usepackage{hyperref}
\begin{document}
\title{}
\author{}
\begin{vdm_al}
--  Overture STANDARD LIBRARY: VDM-Unit
--      --------------------------------------------
-- Version 2.1.0 
--
-- Standard library for the Overture Interpreter. When the interpreter
-- evaluates the preliminary functions/operations in this file,
-- corresponding internal functions is called instead of issuing a run
-- time error. Signatures should not be changed, as well as name of
-- module (VDM-SL) or class (VDM++). Pre/post conditions is 
-- fully user customisable. 
-- Dont care's may NOT be used in the parameter lists.
--  
--
-- define VDM++ equivalents to the JUnit 3.8.2 classes
--
--===========================================================================
-- Using VDMUnit
--
-- VDM Unit generally supports the same features as JUnit for Java
--
-- Approach:
-- * Define one Test Case per class which should be tested
-- ** Specify all test *operations* in the test case
-- * Define a TestResult and TestSuite to execute the tests
--
-- Running the tests:
-- Option 1.
-- Use the automated reflection search to find Test classes.
--  new TestRunner().run()
--
-- Option 2.
-- Use the automated reflection search to find test operations.
--  All operations must begin with "test".
--  Usage:
--      let tests : set of Test = {new TestCase1(), new TestCase2()},
--          ts : TestSuite = new TestSuite(tests),
--          result : TestResult = new TestResult()
--      in
--      (
--          ts.run(result);
--          IO`print(result.toString());
--      );
--
-- Option 3.
-- Use explicit definition to call tests.
--  All test cases must *override* the operation 
--      "protected runTest : () ==> ()"
--  This means that only a single test can be performed in each test case
--
-- Usage:
--      let ts : TestSuite = new TestSuite(),
--          result = new TestResult() 
--      in
--      (
--          ts.addTest(new TestCase1());
--          ts.addTest(new TestCase2());
--          ts.run(result);
--          IO`println(result.toString());
--      )
--
--===========================================================================



-- forward definition of java.lang.Throwable
-- do not code generate (but import from Java library)
class Throwable
end Throwable

-- forward definition of java.lang.Error
-- do not code generate (but import from Java library)
class Error is subclass of Throwable

instance variables
  protected fMessage : seq of char := (*@\vdmnotcovered{}@*)[]
  
operations
  public Error: () ==> Error
  Error () == (*@\vdmnotcovered{ski}@*)p;
  
  public Error: seq of char ==> Error
  Error (pmessage) == (*@\vdmnotcovered{fMessag}@*)e := (*@\vdmnotcovered{pmessag}@*)e;

  public hasMessage: () ==> bool
  hasMessage () == (*@\vdmnotcovered{retur}@*)n (*@\vdmnotcovered{le}@*)n (*@\vdmnotcovered{fMessag}@*)e (*@\vdmnotcovered{}@*)> (*@\vdmnotcovered{}@*)0;
    
  public getMessage: () ==> seq of char
  getMessage () == (*@\vdmnotcovered{retur}@*)n (*@\vdmnotcovered{fMessag}@*)e
    pre (*@\vdmnotcovered{le}@*)n (*@\vdmnotcovered{fMessag}@*)e (*@\vdmnotcovered{}@*)> (*@\vdmnotcovered{}@*)0;--hasMessage()


  public static throw: seq of char ==> ()
  throw (pmessage) == (*@\vdmnotcovered{exi}@*)t (*@\vdmnotcovered{ne}@*)w (*@\vdmnotcovered{Erro}@*)r((*@\vdmnotcovered{pmessag}@*)e)
end Error

--
-- forward definition of junit.framework.AssertionFailedError
--

class AssertionFailedError is subclass of Error

operations
  public AssertionFailedError: () ==> AssertionFailedError
  AssertionFailedError () == (*@\vdmnotcovered{ski}@*)p;
  
  public AssertionFailedError: seq of char ==> AssertionFailedError
  AssertionFailedError (pmessage) == (*@\vdmnotcovered{fMessag}@*)e := (*@\vdmnotcovered{pmessag}@*)e;
   
end AssertionFailedError

--
-- forward definition of junit.framework.Assert
--

class Assert

operations
  public static assertTrue: bool ==> ()
  assertTrue (pbool) ==
    (*@\vdmnotcovered{i}@*)f (*@\vdmnotcovered{no}@*)t (*@\vdmnotcovered{pboo}@*)l then (*@\vdmnotcovered{exi}@*)t (*@\vdmnotcovered{ne}@*)w (*@\vdmnotcovered{AssertionFailedErro}@*)r();
  
  public static assertTrue: seq of char * bool ==> ()
  assertTrue (pmessage, pbool) ==
    (*@\vdmnotcovered{i}@*)f (*@\vdmnotcovered{no}@*)t (*@\vdmnotcovered{pboo}@*)l then (*@\vdmnotcovered{exi}@*)t (*@\vdmnotcovered{ne}@*)w (*@\vdmnotcovered{AssertionFailedErro}@*)r((*@\vdmnotcovered{pmessag}@*)e);
    
  public static assertFalse: bool ==> ()
  assertFalse (pbool) ==
    (*@\vdmnotcovered{i}@*)f (*@\vdmnotcovered{pboo}@*)l then (*@\vdmnotcovered{exi}@*)t (*@\vdmnotcovered{ne}@*)w (*@\vdmnotcovered{AssertionFailedErro}@*)r();
  
  public static assertFalse: seq of char * bool ==> ()
  assertFalse (pmessage, pbool) ==
    (*@\vdmnotcovered{i}@*)f (*@\vdmnotcovered{pboo}@*)l then (*@\vdmnotcovered{exi}@*)t (*@\vdmnotcovered{ne}@*)w (*@\vdmnotcovered{AssertionFailedErro}@*)r((*@\vdmnotcovered{pmessag}@*)e);
    
  public static fail: () ==> ()
  fail () == (*@\vdmnotcovered{exi}@*)t (*@\vdmnotcovered{ne}@*)w (*@\vdmnotcovered{AssertionFailedErro}@*)r();
  
  public static fail: seq of char ==> ()
  fail (pmessage) == (*@\vdmnotcovered{exi}@*)t (*@\vdmnotcovered{ne}@*)w (*@\vdmnotcovered{AssertionFailedErro}@*)r((*@\vdmnotcovered{pmessag}@*)e)
    
end Assert

--
-- forward definition of junit.framework.Test
--

class Test is subclass of Assert

instance variables
  private fName : seq of char := (*@\vdmnotcovered{}@*)[]
  
operations
  -- instance variables access operations
  public getName: () ==> seq of char
  getName () == (*@\vdmnotcovered{retur}@*)n (*@\vdmnotcovered{fNam}@*)e;
  
  public setName: seq of char ==> ()
  setName (name) == (*@\vdmnotcovered{fNam}@*)e := (*@\vdmnotcovered{nam}@*)e;
  
  public run: TestResult ==> ()
  run (-) == is subclass responsibility;
  
  public runOnly: seq of char * TestResult ==> ()
  runOnly (-,-) == is subclass responsibility
    
end Test

--
-- forward definition of junit.framework.TestCase
--

class TestCase is subclass of Test

operations

--  /**
--   * A convenience method to run this test, collecting the results with a
--   * default TestResult object.
--   *
--   * @see TestResult
--   */
    public run :() ==> TestResult
    run() ==
    (
        (*@\vdmnotcovered{le}@*)t result : TestResult = (*@\vdmnotcovered{createResul}@*)t() in (
        (*@\vdmnotcovered{ru}@*)n((*@\vdmnotcovered{resul}@*)t);
        (*@\vdmnotcovered{retur}@*)n (*@\vdmnotcovered{resul}@*)t);
    );
--  /**
--   * Runs the test case and collects the results in TestResult.
--   */
    public run : TestResult ==> ()
    run(result) ==
    (
        (*@\vdmnotcovered{resul}@*)t.run((*@\vdmnotcovered{sel}@*)f);
    );
 
 public runOnly: seq of char * TestResult ==> ()
 runOnly (name,result) ==
 (*@\vdmnotcovered{i}@*)f (*@\vdmnotcovered{nam}@*)e (*@\vdmnotcovered{}@*)= (*@\vdmnotcovered{getNam}@*)e() then
  (*@\vdmnotcovered{resul}@*)t.run((*@\vdmnotcovered{sel}@*)f);
  
    protected createResult : () ==> TestResult
    createResult() == (*@\vdmnotcovered{retur}@*)n (*@\vdmnotcovered{ne}@*)w (*@\vdmnotcovered{TestResul}@*)t();
 
operations
  -- constructor
  public TestCase: seq of char ==> TestCase
  TestCase (name) == (*@\vdmnotcovered{setNam}@*)e((*@\vdmnotcovered{nam}@*)e);

  public TestCase: () ==> TestCase
  TestCase () == (*@\vdmnotcovered{ski}@*)p;
    
  -- template method pattern
  public setUp: () ==> ()
  setUp () == (*@\vdmnotcovered{ski}@*)p;
  
  --
  -- This method might be overriden in a sub class or it will
  --  use reflection to call the test method as specified in the name
  --
  protected runTest: () ==> ()
  runTest () == (*@\vdmnotcovered{reflectionRunTes}@*)t((*@\vdmnotcovered{sel}@*)f,(*@\vdmnotcovered{getNam}@*)e());

  --
  -- External Method
  --
  private static reflectionRunTest : TestCase * seq of char ==> ()
  reflectionRunTest(obj,name) == is not yet specified;
     
  public runBare : () ==> ()
  runBare() ==
  (
    dcl exception : [Throwable] := (*@\vdmnotcovered{ni}@*)l;
    
    (*@\vdmnotcovered{setU}@*)p();
    
    (*@\vdmnotcovered{tra}@*)p exc :Throwable
        with  
            (*@\vdmnotcovered{exceptio}@*)n := (*@\vdmnotcovered{ex}@*)c
        in
        (
            (*@\vdmnotcovered{runTes}@*)t();
        );

    (*@\vdmnotcovered{tra}@*)p exc :Throwable
        with  
            (*@\vdmnotcovered{i}@*)f (*@\vdmnotcovered{exceptio}@*)n (*@\vdmnotcovered{}@*)= (*@\vdmnotcovered{ni}@*)l then
                (*@\vdmnotcovered{exceptio}@*)n := (*@\vdmnotcovered{ex}@*)c
        in
        (
            (*@\vdmnotcovered{tearDow}@*)n();
        );
        
    (*@\vdmnotcovered{i}@*)f (*@\vdmnotcovered{exceptio}@*)n (*@\vdmnotcovered{<}@*)> (*@\vdmnotcovered{ni}@*)l then (*@\vdmnotcovered{exi}@*)t (*@\vdmnotcovered{exceptio}@*)n;
   );
  
  public tearDown: () ==> ()
  tearDown () == (*@\vdmnotcovered{ski}@*)p;
           
end TestCase

--
-- forward definition of junit.framework.TestSuite
--

class TestSuite is subclass of Test

instance variables
  private fTests : seq of Test := (*@\vdmnotcovered{}@*)[]
  
operations
  public TestSuite: seq of char ==> TestSuite
  TestSuite (name) == (*@\vdmnotcovered{setNam}@*)e((*@\vdmnotcovered{nam}@*)e);

  public TestSuite: () ==> TestSuite
  TestSuite () == (*@\vdmnotcovered{setNam}@*)e((*@\vdmnotcovered{"}@*)");

  public TestSuite: set of Test * seq of char ==> TestSuite
  TestSuite (tests,name) == 
  (
    (*@\vdmnotcovered{setNam}@*)e((*@\vdmnotcovered{nam}@*)e);
    (*@\vdmnotcovered{fo}@*)r all test in set (*@\vdmnotcovered{test}@*)s do
    (
       (*@\vdmnotcovered{fo}@*)r all t in set (*@\vdmnotcovered{elem}@*)s (*@\vdmnotcovered{createTest}@*)s((*@\vdmnotcovered{tes}@*)t) do
       (
         (*@\vdmnotcovered{addTes}@*)t((*@\vdmnotcovered{}@*)t); 
       );
    );
    
  );

  public TestSuite: set of Test ==> TestSuite
  TestSuite (tests) == 
  (
    (*@\vdmnotcovered{setNam}@*)e((*@\vdmnotcovered{"}@*)");
    (*@\vdmnotcovered{fo}@*)r all test in set (*@\vdmnotcovered{test}@*)s do
    (
       (*@\vdmnotcovered{fo}@*)r all t in set (*@\vdmnotcovered{elem}@*)s (*@\vdmnotcovered{createTest}@*)s((*@\vdmnotcovered{tes}@*)t) do
       (
         (*@\vdmnotcovered{addTes}@*)t((*@\vdmnotcovered{}@*)t); 
       );
    );
    
  );

  public TestSuite: Test ==> TestSuite
  TestSuite (test) == 
  (
    (*@\vdmnotcovered{setNam}@*)e((*@\vdmnotcovered{"}@*)");
    (*@\vdmnotcovered{fo}@*)r all t in set (*@\vdmnotcovered{elem}@*)s (*@\vdmnotcovered{createTest}@*)s((*@\vdmnotcovered{tes}@*)t) do
     (
       (*@\vdmnotcovered{addTes}@*)t((*@\vdmnotcovered{}@*)t); 
     );
  );

  public TestSuite: Test * seq of char ==> TestSuite
  TestSuite (test,name) == 
  (
    (*@\vdmnotcovered{setNam}@*)e((*@\vdmnotcovered{nam}@*)e);
    (*@\vdmnotcovered{fo}@*)r all t in set (*@\vdmnotcovered{elem}@*)s (*@\vdmnotcovered{createTest}@*)s((*@\vdmnotcovered{tes}@*)t) do
     (
       (*@\vdmnotcovered{addTes}@*)t((*@\vdmnotcovered{}@*)t); 
     );
  );
  
  public addTest: Test ==> ()
  addTest (ptst) == (*@\vdmnotcovered{fTest}@*)s := (*@\vdmnotcovered{fTest}@*)s (*@\vdmnotcovered{}@*)^ (*@\vdmnotcovered{}@*)[(*@\vdmnotcovered{pts}@*)t];
  
  -- implementing the abstract run operation 
  public run: TestResult ==> ()
  run (ptr) ==(*@\vdmnotcovered{fo}@*)r ptst in (*@\vdmnotcovered{fTest}@*)s do (*@\vdmnotcovered{pts}@*)t.run((*@\vdmnotcovered{pt}@*)r);

  public runOnly: seq of char * TestResult ==> ()
  runOnly (name,ptr) ==
  (*@\vdmnotcovered{fo}@*)r ptst in (*@\vdmnotcovered{fTest}@*)s do
  (*@\vdmnotcovered{i}@*)f (*@\vdmnotcovered{pts}@*)t.(*@\vdmnotcovered{getNam}@*)e() (*@\vdmnotcovered{}@*)= (*@\vdmnotcovered{nam}@*)e then
 (*@\vdmnotcovered{pts}@*)t.run((*@\vdmnotcovered{pt}@*)r);
  
  public run: Test * TestResult ==> ()
  run (test, ptr) == (*@\vdmnotcovered{tes}@*)t.run((*@\vdmnotcovered{pt}@*)r);
      
    public tests : () ==> seq of Test
    tests()==(*@\vdmnotcovered{retur}@*)n (*@\vdmnotcovered{fTest}@*)s;
    
    public testCount : () ==> int
    testCount()== (*@\vdmnotcovered{retur}@*)n (*@\vdmnotcovered{le}@*)n (*@\vdmnotcovered{fTest}@*)s;
    
    public testAt : int ==> Test
    testAt(index) == (*@\vdmnotcovered{retur}@*)n (*@\vdmnotcovered{fTest}@*)s((*@\vdmnotcovered{inde}@*)x)
    pre (*@\vdmnotcovered{inde}@*)x (*@\vdmnotcovered{}@*)>(*@\vdmnotcovered{}@*)0 (*@\vdmnotcovered{an}@*)d (*@\vdmnotcovered{inde}@*)x (*@\vdmnotcovered{}@*)< (*@\vdmnotcovered{le}@*)n (*@\vdmnotcovered{fTest}@*)s;


  private getTestMethodNamed : Test ==> seq of seq of char
  getTestMethodNamed(test)== is not yet specified;

  public createTests : Test ==> seq of Test
  createTests(test)== is not yet specified;


end TestSuite

--
-- forward definition of junit.framework.TestListener
--

class TestListener

operations
  public initListener: () ==> ()
  initListener () == is subclass responsibility;
  
  public exitListener: () ==> ()
  exitListener () == is subclass responsibility;
  
  public addFailure: Test * AssertionFailedError ==> ()
  addFailure (-, -) == is subclass responsibility; 

  public addError: Test * Throwable ==> ()
  addError (-, -) == is subclass responsibility;
  
  public startTest: Test ==> ()
  startTest (-) == is subclass responsibility;
  
  public endTest: Test ==> ()
  endTest (-) == is subclass responsibility;
  
end TestListener

--
-- forward definition of junit.framework.TestResult
--

class TestResult

types
  private InternalError :: tcname : seq of char
                           except : Error
                   
instance variables
  -- keep track of assertions failed
  private fFailures : seq of AssertionFailedError := (*@\vdmnotcovered{}@*)[];
  
  -- keep track of exceptions thrown
  private fErrors : seq of InternalError := (*@\vdmnotcovered{}@*)[];
  
  -- count the number of executed tests
  private fRunTests : nat := (*@\vdmnotcovered{}@*)0;
  
  -- the list of interested listeners
  private fListeners : set of TestListener := (*@\vdmnotcovered{}@*){}
  
operations
  public addListener: TestListener ==> ()
  addListener (ptl) == 
    ( -- first add the listener to the set
      (*@\vdmnotcovered{fListener}@*)s := (*@\vdmnotcovered{fListener}@*)s (*@\vdmnotcovered{unio}@*)n (*@\vdmnotcovered{}@*){(*@\vdmnotcovered{pt}@*)l};
      -- execute the initializer for the listener
      (*@\vdmnotcovered{pt}@*)l.initListener() )
    pre (*@\vdmnotcovered{pt}@*)l (*@\vdmnotcovered{no}@*)t in set (*@\vdmnotcovered{fListener}@*)s;
  
  public removeListener: TestListener ==> ()
  removeListener (ptl) ==
    ( -- execute the exit operation for the listener
      (*@\vdmnotcovered{pt}@*)l.exitListener();
      -- remove the listener from the set
      (*@\vdmnotcovered{fListener}@*)s := (*@\vdmnotcovered{fListener}@*)s (*@\vdmnotcovered{}@*)\ (*@\vdmnotcovered{}@*){(*@\vdmnotcovered{pt}@*)l} )
    pre (*@\vdmnotcovered{pt}@*)l (*@\vdmnotcovered{i}@*)n set (*@\vdmnotcovered{fListener}@*)s;
  
  public addFailure: Test * AssertionFailedError ==> ()
  addFailure (ptst, pafe) == 
    ( -- first handle the failed assertion locally
      (*@\vdmnotcovered{i}@*)f (*@\vdmnotcovered{paf}@*)e.(*@\vdmnotcovered{hasMessag}@*)e()
      then (*@\vdmnotcovered{fFailure}@*)s := (*@\vdmnotcovered{fFailure}@*)s (*@\vdmnotcovered{}@*)^ (*@\vdmnotcovered{}@*)[(*@\vdmnotcovered{paf}@*)e]
      else (*@\vdmnotcovered{le}@*)t str = (*@\vdmnotcovered{"Assertion failed in test }@*)"(*@\vdmnotcovered{}@*)^(*@\vdmnotcovered{pts}@*)t.(*@\vdmnotcovered{getNam}@*)e() in
             (*@\vdmnotcovered{fFailure}@*)s := (*@\vdmnotcovered{fFailure}@*)s (*@\vdmnotcovered{}@*)^ (*@\vdmnotcovered{}@*)[(*@\vdmnotcovered{ne}@*)w (*@\vdmnotcovered{AssertionFailedErro}@*)r((*@\vdmnotcovered{st}@*)r)];
      -- now inform all listeners
      (*@\vdmnotcovered{fo}@*)r all listener in set (*@\vdmnotcovered{fListener}@*)s do
        (*@\vdmnotcovered{listene}@*)r.addFailure((*@\vdmnotcovered{pts}@*)t, (*@\vdmnotcovered{paf}@*)e) );
  
  public addError: Test * Throwable ==> ()
  addError (ptst, perr) ==
    ( -- add the error to the list of exceptions caught so far
      (*@\vdmnotcovered{fError}@*)s := (*@\vdmnotcovered{fError}@*)s (*@\vdmnotcovered{}@*)^ (*@\vdmnotcovered{}@*)[(*@\vdmnotcovered{mk\_InternalErro}@*)r((*@\vdmnotcovered{pts}@*)t.(*@\vdmnotcovered{getNam}@*)e(), (*@\vdmnotcovered{per}@*)r)];
      -- now inform all listeners
      (*@\vdmnotcovered{fo}@*)r all listener in set (*@\vdmnotcovered{fListener}@*)s do
        (*@\vdmnotcovered{listene}@*)r.addError((*@\vdmnotcovered{pts}@*)t, (*@\vdmnotcovered{per}@*)r) );

  public startTest: Test ==> ()
  startTest (ptst) ==
    ( -- first increase the local run test counter
      (*@\vdmnotcovered{fRunTest}@*)s := (*@\vdmnotcovered{fRunTest}@*)s (*@\vdmnotcovered{}@*)+ (*@\vdmnotcovered{}@*)1;
      -- now inform all listeners
      (*@\vdmnotcovered{fo}@*)r all listener in set (*@\vdmnotcovered{fListener}@*)s do
        (*@\vdmnotcovered{listene}@*)r.startTest((*@\vdmnotcovered{pts}@*)t) );
  
  public endTest: Test ==> ()
  endTest (ptst) ==
    (*@\vdmnotcovered{fo}@*)r all listener in set (*@\vdmnotcovered{fListener}@*)s do
      (*@\vdmnotcovered{listene}@*)r.endTest((*@\vdmnotcovered{pts}@*)t);

  public run : TestCase ==> ()
  run(test)==
  ( 
    (*@\vdmnotcovered{startTes}@*)t((*@\vdmnotcovered{tes}@*)t);
    (*@\vdmnotcovered{runProtecte}@*)d((*@\vdmnotcovered{tes}@*)t);
    (*@\vdmnotcovered{endTes}@*)t((*@\vdmnotcovered{tes}@*)t);

  );
  public runProtected : TestCase ==> () 
  runProtected(test) ==
  (
    (*@\vdmnotcovered{tra}@*)p exc :Throwable
        with
          (*@\vdmnotcovered{i}@*)f (*@\vdmnotcovered{isofclas}@*)s(AssertionFailedError,(*@\vdmnotcovered{ex}@*)c)
          then (*@\vdmnotcovered{addFailur}@*)e((*@\vdmnotcovered{tes}@*)t, (*@\vdmnotcovered{ex}@*)c)
          else (*@\vdmnotcovered{i}@*)f (*@\vdmnotcovered{isofbaseclas}@*)s(Throwable, (*@\vdmnotcovered{ex}@*)c)
               then (*@\vdmnotcovered{addErro}@*)r((*@\vdmnotcovered{tes}@*)t, (*@\vdmnotcovered{ex}@*)c)
               else (*@\vdmnotcovered{erro}@*)r -- We hope this never happens
        in (
              (*@\vdmnotcovered{tes}@*)t.runBare();
            ); 
  );
  
  public runCount: () ==> nat
  runCount () == (*@\vdmnotcovered{retur}@*)n (*@\vdmnotcovered{fRunTest}@*)s;
    
  public failureCount: () ==> nat
  failureCount () == (*@\vdmnotcovered{retur}@*)n (*@\vdmnotcovered{le}@*)n (*@\vdmnotcovered{fFailure}@*)s;
  
  public errorCount: () ==> nat
  errorCount () == (*@\vdmnotcovered{retur}@*)n (*@\vdmnotcovered{le}@*)n (*@\vdmnotcovered{fError}@*)s;
  
  public wasSuccessful: () ==> bool
  wasSuccessful () == (*@\vdmnotcovered{retur}@*)n (*@\vdmnotcovered{failureCoun}@*)t() (*@\vdmnotcovered{}@*)= (*@\vdmnotcovered{}@*)0 (*@\vdmnotcovered{an}@*)d (*@\vdmnotcovered{errorCoun}@*)t() (*@\vdmnotcovered{}@*)= (*@\vdmnotcovered{}@*)0;

  public toString : () ==> seq of char
  toString()==
  (
    dcl tmp: seq of char :=     
        (*@\vdmnotcovered{"----------------------------------------\textbackslash n}@*)"(*@\vdmnotcovered{}@*)^
        (*@\vdmnotcovered{"|    TEST RESULTS                      |\textbackslash n}@*)"(*@\vdmnotcovered{}@*)^
        (*@\vdmnotcovered{"|--------------------------------------|\textbackslash n}@*)"(*@\vdmnotcovered{}@*)^
        (*@\vdmnotcovered{"| Executed: }@*)"(*@\vdmnotcovered{}@*)^(*@\vdmnotcovered{ToStringIn}@*)t((*@\vdmnotcovered{runCoun}@*)t())(*@\vdmnotcovered{}@*)^(*@\vdmnotcovered{"\textbackslash n}@*)"(*@\vdmnotcovered{}@*)^
        (*@\vdmnotcovered{"| Failures: }@*)"(*@\vdmnotcovered{}@*)^(*@\vdmnotcovered{ToStringIn}@*)t((*@\vdmnotcovered{failureCoun}@*)t())(*@\vdmnotcovered{}@*)^(*@\vdmnotcovered{"\textbackslash n}@*)"(*@\vdmnotcovered{}@*)^
        (*@\vdmnotcovered{"| Errors:   }@*)"(*@\vdmnotcovered{}@*)^(*@\vdmnotcovered{ToStringIn}@*)t((*@\vdmnotcovered{errorCoun}@*)t())(*@\vdmnotcovered{}@*)^(*@\vdmnotcovered{"\textbackslash n}@*)";
    
    (*@\vdmnotcovered{tm}@*)p := (*@\vdmnotcovered{tm}@*)p(*@\vdmnotcovered{}@*)^(*@\vdmnotcovered{"|\_\_\_\_\_\_\_\_\_\_\_\_\_\_\_\_\_\_\_\_\_\_\_\_\_\_\_\_\_\_\_\_\_\_\_\_\_\_|\textbackslash n|}@*)";
    (*@\vdmnotcovered{fo}@*)r all i in set (*@\vdmnotcovered{ind}@*)s (*@\vdmnotcovered{fFailure}@*)s do
        (*@\vdmnotcovered{tm}@*)p := (*@\vdmnotcovered{tm}@*)p(*@\vdmnotcovered{}@*)^(*@\vdmnotcovered{"\textbackslash n| }@*)"(*@\vdmnotcovered{}@*)^(*@\vdmnotcovered{fFailure}@*)s((*@\vdmnotcovered{}@*)i).(*@\vdmnotcovered{getMessag}@*)e();
    
    (*@\vdmnotcovered{tm}@*)p := (*@\vdmnotcovered{tm}@*)p(*@\vdmnotcovered{}@*)^(*@\vdmnotcovered{"\textbackslash n|}@*)";

    (*@\vdmnotcovered{fo}@*)r all i in set (*@\vdmnotcovered{ind}@*)s (*@\vdmnotcovered{fError}@*)s do
        (*@\vdmnotcovered{tm}@*)p := (*@\vdmnotcovered{tm}@*)p(*@\vdmnotcovered{}@*)^(*@\vdmnotcovered{"\textbackslash n| Error in test }@*)"(*@\vdmnotcovered{}@*)^(*@\vdmnotcovered{fError}@*)s((*@\vdmnotcovered{}@*)i).(*@\vdmnotcovered{tcnam}@*)e(*@\vdmnotcovered{}@*)^(*@\vdmnotcovered{" }@*)"(*@\vdmnotcovered{}@*)^(*@\vdmnotcovered{fError}@*)s((*@\vdmnotcovered{}@*)i).(*@\vdmnotcovered{excep}@*)t.(*@\vdmnotcovered{getMessag}@*)e();
 
    (*@\vdmnotcovered{tm}@*)p := (*@\vdmnotcovered{tm}@*)p(*@\vdmnotcovered{}@*)^(*@\vdmnotcovered{"\textbackslash n|--------------------------------------|\textbackslash n}@*)";

    (*@\vdmnotcovered{i}@*)f (*@\vdmnotcovered{wasSuccessfu}@*)l() 
    then
    (
        (*@\vdmnotcovered{tm}@*)p := (*@\vdmnotcovered{tm}@*)p(*@\vdmnotcovered{}@*)^(*@\vdmnotcovered{"|              SUCCESS                 |}@*)";
    )
    else
    (
        (*@\vdmnotcovered{tm}@*)p := (*@\vdmnotcovered{tm}@*)p(*@\vdmnotcovered{}@*)^(*@\vdmnotcovered{"|              FAILURE                 |}@*)";
    );
        (*@\vdmnotcovered{tm}@*)p := (*@\vdmnotcovered{tm}@*)p(*@\vdmnotcovered{}@*)^(*@\vdmnotcovered{"\textbackslash n|\_\_\_\_\_\_\_\_\_\_\_\_\_\_\_\_\_\_\_\_\_\_\_\_\_\_\_\_\_\_\_\_\_\_\_\_\_\_|}@*)";
        (*@\vdmnotcovered{retur}@*)n (*@\vdmnotcovered{tm}@*)p;
  );
functions
-------------------------------------
-- Convert a int to a String
-------------------------------------
public static ToStringInt : int | nat1 | nat -> seq of char
ToStringInt(val) ==
  (*@\vdmnotcovered{le}@*)t result : int = (*@\vdmnotcovered{va}@*)l (*@\vdmnotcovered{mo}@*)d (*@\vdmnotcovered{1}@*)0,
      rest : int = (*@\vdmnotcovered{va}@*)l (*@\vdmnotcovered{di}@*)v (*@\vdmnotcovered{1}@*)0
  in (*@\vdmnotcovered{i}@*)f (*@\vdmnotcovered{res}@*)t (*@\vdmnotcovered{}@*)> (*@\vdmnotcovered{}@*)0 then
       (*@\vdmnotcovered{ToStringIn}@*)t((*@\vdmnotcovered{res}@*)t) (*@\vdmnotcovered{}@*)^ (*@\vdmnotcovered{GetStringFromNu}@*)m((*@\vdmnotcovered{resul}@*)t)
     else (*@\vdmnotcovered{GetStringFromNu}@*)m((*@\vdmnotcovered{resul}@*)t)
pre (*@\vdmnotcovered{va}@*)l (*@\vdmnotcovered{>}@*)= (*@\vdmnotcovered{}@*)0
measure ToStringIntMeasure;

private ToStringIntMeasure : nat -> nat
ToStringIntMeasure(a) == (*@\vdmnotcovered{}@*)a;
    
-------------------------------------
-- Convert a int < 10 to a String
-------------------------------------
public static GetStringFromNum : int  -> seq of char --| nat | real
GetStringFromNum(val) == (*@\vdmnotcovered{}@*)[(*@\vdmnotcovered{"0123456789}@*)"((*@\vdmnotcovered{va}@*)l(*@\vdmnotcovered{}@*)+(*@\vdmnotcovered{}@*)1)]
pre (*@\vdmnotcovered{va}@*)l (*@\vdmnotcovered{}@*)< (*@\vdmnotcovered{1}@*)0;
  
end TestResult  



class TestRunner

operations
public run : () ==> ()
run() ==
      (*@\vdmnotcovered{le}@*)t tests : set of Test = (*@\vdmnotcovered{collectTest}@*)s((*@\vdmnotcovered{ne}@*)w (*@\vdmnotcovered{TestRunne}@*)r()),
          ts : TestSuite = (*@\vdmnotcovered{ne}@*)w (*@\vdmnotcovered{TestSuit}@*)e((*@\vdmnotcovered{test}@*)s),
          result : TestResult = (*@\vdmnotcovered{ne}@*)w (*@\vdmnotcovered{TestResul}@*)t()
      in
      (
          (*@\vdmnotcovered{t}@*)s.run((*@\vdmnotcovered{resul}@*)t);
          (*@\vdmnotcovered{IO`prin}@*)t((*@\vdmnotcovered{resul}@*)t.(*@\vdmnotcovered{toStrin}@*)g());
      );

collectTests : TestRunner ==> set of Test
collectTests(obj) == is not yet specified;

end TestRunner
\end{vdm_al}
\bigskip
\begin{longtable}{|l|r|r|}
\hline
Function or operation & Coverage & Calls \\
\hline
\hline
Assert & 0.0\% & 0 \\
\hline
AssertionFailedError & 0.0\% & 0 \\
\hline
AssertionFailedError & 0.0\% & 0 \\
\hline
AssertionFailedError:106 & 0.0\% & 0 \\
\hline
Error & 0.0\% & 0 \\
\hline
Error & 0.0\% & 0 \\
\hline
Error:81 & 0.0\% & 0 \\
\hline
GetStringFromNum & 0.0\% & 0 \\
\hline
Test & 28.5\% & 0 \\
\hline
TestCase & 0.0\% & 0 \\
\hline
TestCase & 0.0\% & 0 \\
\hline
TestCase:210 & 0.0\% & 0 \\
\hline
TestListener & 100.0\% & 1 \\
\hline
TestResult & 0.0\% & 0 \\
\hline
TestRunner & 0.0\% & 0 \\
\hline
TestSuite & 0.0\% & 0 \\
\hline
TestSuite & 0.0\% & 0 \\
\hline
TestSuite:275 & 0.0\% & 0 \\
\hline
TestSuite:278 & 0.0\% & 0 \\
\hline
TestSuite:292 & 0.0\% & 0 \\
\hline
TestSuite:306 & 0.0\% & 0 \\
\hline
TestSuite:316 & 0.0\% & 0 \\
\hline
Throwable & 0.0\% & 0 \\
\hline
ToStringInt & 0.0\% & 0 \\
\hline
ToStringIntMeasure & 0.0\% & 0 \\
\hline
addError & 100.0\% & 1 \\
\hline
addError & 0.0\% & 0 \\
\hline
addFailure & 100.0\% & 1 \\
\hline
addFailure & 0.0\% & 0 \\
\hline
addListener & 0.0\% & 0 \\
\hline
addTest & 0.0\% & 0 \\
\hline
assertFalse & 0.0\% & 0 \\
\hline
assertFalse:130 & 0.0\% & 0 \\
\hline
assertTrue & 0.0\% & 0 \\
\hline
assertTrue:122 & 0.0\% & 0 \\
\hline
collectTests & 0.0\% & 0 \\
\hline
createResult & 0.0\% & 0 \\
\hline
createTests & 0.0\% & 0 \\
\hline
endTest & 0.0\% & 0 \\
\hline
endTest & 100.0\% & 1 \\
\hline
errorCount & 0.0\% & 0 \\
\hline
exitListener & 100.0\% & 1 \\
\hline
fail & 0.0\% & 0 \\
\hline
fail:137 & 0.0\% & 0 \\
\hline
failureCount & 0.0\% & 0 \\
\hline
getMessage & 0.0\% & 0 \\
\hline
getName & 0.0\% & 0 \\
\hline
getTestMethodNamed & 0.0\% & 0 \\
\hline
hasMessage & 0.0\% & 0 \\
\hline
initListener & 100.0\% & 1 \\
\hline
reflectionRunTest & 0.0\% & 0 \\
\hline
removeListener & 0.0\% & 0 \\
\hline
run & 0.0\% & 0 \\
\hline
run & 0.0\% & 0 \\
\hline
run & 100.0\% & 1 \\
\hline
run & 0.0\% & 0 \\
\hline
run & 0.0\% & 0 \\
\hline
run:191 & 0.0\% & 0 \\
\hline
run:339 & 0.0\% & 0 \\
\hline
runBare & 0.0\% & 0 \\
\hline
runCount & 0.0\% & 0 \\
\hline
runOnly & 0.0\% & 0 \\
\hline
runOnly & 0.0\% & 0 \\
\hline
runOnly & 100.0\% & 1 \\
\hline
runProtected & 0.0\% & 0 \\
\hline
runTest & 0.0\% & 0 \\
\hline
setName & 0.0\% & 0 \\
\hline
setUp & 0.0\% & 0 \\
\hline
startTest & 100.0\% & 1 \\
\hline
startTest & 0.0\% & 0 \\
\hline
tearDown & 0.0\% & 0 \\
\hline
testAt & 0.0\% & 0 \\
\hline
testCount & 0.0\% & 0 \\
\hline
tests & 0.0\% & 0 \\
\hline
throw & 0.0\% & 0 \\
\hline
toString & 0.0\% & 0 \\
\hline
wasSuccessful & 0.0\% & 0 \\
\hline
\hline
VDMUnit.vdmpp & 1.8\% & 9 \\
\hline
\end{longtable}
\end{document}
